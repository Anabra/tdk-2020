\documentclass[main.tex]{subfiles}
\begin{document}
	
	Over the last few years, the functional programming paradigm has become even more popular and prominent than it was before. More and more industrial applications emerge, the paradigm itself keeps evolving, existing functional languages are being refined day by day, and even completely new languages appear. Yet, it seems the corresponding compiler technology lacks behind a bit.
	
	\section{Motivation}
	
	Functional languages come with a multitude of interesting features that allow us to write programs on higher abstraction levels. Some of these features include higher-order functions, laziness and sophisticated type systems based on SystemFC~\cite{systemfc}, some even supporting dependent types. Although these features make writing code more convenient, they also complicate the compilation process.
	
	Compiler front ends usually handle these problems very well, but the back ends often struggle to produce efficient low level code. The reason for this is that back ends have a hard time optimizing code containing \emph{functional artifacts}. These functional artifacts are the by-products of high-level language features mentioned earlier. For example, higher-order functions can introduce unknown function calls and laziness can result in implicit value evaluation which can prove to be very hard to optimize. As a consequence, compilers generally compromise on low level efficiency for high-level language features.
	
	Moreover, the paradigm itself also encourages a certain programming style which further complicates the situation. Functional code usually consists of many smaller functions, rather than fewer big ones. This style of coding results in more composable programs, but also presents more difficulties for compilation, since optimizing individual functions only is no longer sufficient. 
	
	In order to resolve these problems, we need a compiler back end that can optimize across functions as well as allow the optimization of laziness in some way. Also, it would be beneficial if the back end could theoretically handle any suitable front end language.
	
	\section{Contributions}
	
	In this thesis we present a modern look at the GRIN framework. We explain some of its core concepts, and also provide a number of improvements to it. The results are demonstrated through a modernized implementation of the framework\footnote{Almost the entire GRIN framework has been reimplemented. The only exceptions are the simplifying transformations which are no longer needed by the new code generator that uses LLVM as its target language}. There are several improvements to the original GRIN framework mentioned in this thesis\footnote{For example the implementation of the LLVM back end, the implementation of the Idris front end and the Datalog model of the new GRIN IR}, however only the following contributions can be attributed to me.
	
	\hspace{0.5cm}
	\begin{enumerate}
		\item Syntactical extension of the original GRIN IR
		
		\item Datalog formalization of interprocedural liveness analysis for the new GRIN IR 
		
		\item Adoption of dead data elimination transformation, and its extension with typed dummification
		
		\item Specification and evaluation of Idris-specific optimization pipeline
	\end{enumerate}

	\section{Structure of the thesis}
	
	In Chapter~\ref{chap:intro} we provide a very high-level overview of functional language compilation, and present our main contributions. In Chapter~\ref{chap:grin} we introduce the reader to a general overview of Graph Reduction Intermediate Notation as it was presented in Urban Boquist's PhD thesis~\cite{boquist-phd}. In Chapter~\ref{chap:extended-syntax} we present a syntactically extended version of the GRIN IR, which is motivated by the Datalog program analyses explained in Chapter~\ref{chap:datalog-grin}. Chapter~\ref{chap:dce} showcases the different dead code eliminating transformations, and discusses the dead data elimination and its extensions in detail. Chapters~\ref{chap:idris-front-end}~and~\ref{chap:llvm} briefly explain the structure of the Idris front end. and the challenges of compiling GRIN to LLVM respectively. Chapter~\ref{chap:results} shows preliminary results of the current framework, while Chapter~\ref{chap:future-work} discusses potential further improvements. In Chapter~\ref{chap:related-work}, we provide some comparisons to our approach of functional language compilation. Finnaly, Chapter~\ref{chap:conclusions} concludes the thesis.
	
\end{document}