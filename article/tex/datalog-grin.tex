\documentclass[main.tex]{subfiles}
\begin{document}

  In this chapter we discuss logic defined program analyses for GRIN. First, we show how the GRIN language can be modeled by mathematical relations. Then, through a standard points-to analysis, we illustrate how program analyses can be defined via recursive relations in Datalog. Finally, --- using the mentioned Datalog model --- we present a logic-based formalization of the created-by and liveness data-flow analyses.

	\section{Motivation}
	
	Static program analysis specifications are usually defined in a declarative manner through mutually recursive relations. Their implementations however, are often given in an imperative style in order to maximize performance. This discrepancy between the denotational and operational semantics leads to very complicated hand-written implementations.
	
	Fortunately, many static program analysis frameworks \cite{doop, cclyzer, souffle-2} already addressed this issue. These tools use Datalog --- a declarative logic programming language --- as the surface language for defining analyses, then the Datalog program is automatically compiled to efficient imperative code. Datalog allows the programmer to define mutually recursive relations which then are evaluated to a fixed point. This approach facilitates the implementation of complex program analyses by allowing the programmer to focus on the high level specification, meanwhile guaranteeing competitive performance.

  \section{Datalog model of GRIN} \label{sec:datalog-model}
  
  In order to define Datalog analyses for GRIN, the first thing we need need is a Datalog model of the language itself. That is, a declarative, relation-based representation of GRIN programs, that captures the same information that an abstract syntax tree would. In fact, it is possible to define a model from which the original program can be recreated. This suggests an isomorphism between the AST and the Datalog model. However, for our purposes a less precise model is sufficient. 
  
  The rules below define the Datalog model of the GRIN IR. The rules are purely syntactic, and consist of a GRIN program snippet and the corresponding Datalog relation(s). The snippets can be thought of as patterns, which are matched against the current node during the traversal of the abstract syntax tree. If they match, then the relation(s) below the line are "emitted". Hence, these rules are called \emph{emission rules}. The emitted relations will serve as the input relations for the analyses.
  
  \subsection{Notation}
  
  Some of the emission rules contain templates, special functions or partial program snippets. Templates are surrounded by angled brackets and they should be interpreted as holes for specific program elements. For example, the template \pilcode{<lit>} marks an arbitrary literal. There are three special functions: $\tau$, $\pi$ and $\epsilon$. The expression $\tau(x)$ denotes the type of $x$, $\pi(x)$ denotes the position (zero-based index) of $x$ and $\epsilon(x)$ denotes whether $x$ is effectful. $\tau$ is used to determine the types of literals, $\pi$ is used to determine the position of an argument or parameter (of a node or function respectively) and $\epsilon$ is used to determine whether an external is side-effecting or not. $\epsilon(pure)$ is false, $\epsilon(effectful)$ is true. Finally, "\dots{}" in partial program snippets are used to denote syntactically valid, but irrelevant parts of the program. In some rules they are used to denote missing arguments or case alternatives, in which case the rule should be applied to all arguments and case alternatives.
  
  Furthermore we use the terminology "function parameter" and "function argument" to mean formal and actual function arguments respectively. Also, "node parameter" means the variable arguments in an @pattern (on the left-hand side), and "node argument" means the variable arguments in a node value (on the right-hand side).
  
  \subsection{Simple instructions}
  
  These rules are responsible for handling the syntactically simple instructions of GRIN. These are literal assignment (\textsc{ER-Lit}), variable copying (\textsc{ER-Move}), store (\textsc{ER-Store}), fetch (\textsc{ER-Fetch}) and update (\textsc{ERF-Update}). The rules can be seen in Figure~\ref{fig:er-simple}. 

  \begin{figure}[h]
    \begin{mathpar}
    \EmissionRule{\LitAssign{k}{\tau(lit)}{lit}}
                 {\pilcode{k <- pure <lit>}}
                 {ER-Lit}
    \and
    \EmissionRule{\Move{y}{x}}
                {\pilcode{y <- pure x}}
                {ER-Move}
    \and
    \EmissionRule{\Store{p}{n}}
                {\pilcode{p <- store n}}
                {ER-Store}
    \and
    \EmissionRule{\Fetch{n}{p}}
                {\pilcode{n <- fetch p}}
                {ER-Fetch}
    \and
    \EmissionRule{\Update{x}{p}{n}}
                {\pilcode{x <- update p n}}
                {ER-Update}
    \end{mathpar}
  \caption{Emission rules for simple instructions}
  \label{fig:er-simple}
  \end{figure}

	\subsection{Function calls and node values}
	
	The rules \textsc{ER-Call}, \textsc{ER-Node} and \textsc{ER-AsPat} handle function calls, node value creation and @patterns respectively. They can be seen in Figure~\ref{fig:er-funcalls-nodevals}.

	\begin{figure}[h]
		\begin{mathpar}
		  \EmissionRule{\Call{r}{f} \\\\
                    \CallArgument{r}{\pi(x)}{x}}
                    {\pilcode{r <- f ... x ...}}
                    {ER-Call}
      \\\\\\
      \EmissionRule{\Node{n}{tag} \\\\
                    \NodeArgument{n}{\pi(x)}{x}}
                    {\pilcode{n <- pure (tag ... x ...)}}
                    {ER-Node}
      \and
      \EmissionRule{\NodePattern{n}{tag}{m} \\\\
                    \NodeParameter{n}{\pi(x)}{x}}
                    {\pilcode{(tag ... x ...) @ n <- pure m}}
                    {ER-AsPat}
		\end{mathpar}
	\caption{Emission rules for function calls and node values}
	\label{fig:er-funcalls-nodevals}
	\end{figure}

	\subsection{Case expressions}
	
	These rules handle case expression. \textsc{ER-Case} relates the case expression to a name; \textsc{ER-AltRet} relates the return value of a binding sequence to the encompassing alternative's name; \textsc{ER-AltLit}, \textsc{ER-AltDef} and \textsc{ER-AltNode} represent the literal, default and node-valued case alternatives. The rules can be seen in Figure~\ref{fig:er-case}.

  \begin{figure}[h]
    \begin{mathpar}
    \EmissionRule{\Case{r}{s}}
                {\pilcode{r <- case s of ...}}
                {ER-Case}
    \and
    \EmissionRuleL{\ReturnValue{alt}{x}}
                  {\pilcode{... @ alt -> } \\\\
                  \\ \dots \\\\
                  \\ \pilcode{pure x}}
                  {ER-AltRet}
    \and
    \EmissionRuleL{\AltLiteral{r}{alt}{lit}}
                {\pilcode{r <- case s of} \\\\
                  \\ \pilcode{<lit> @ alt -> ...}}
                {ER-AltLit}
    \and
    \EmissionRuleL{\AltDefault{r}{alt}}
                {\pilcode{r <- case s of} \\\\
                  \\ \pilcode{\#default @ alt -> ...}}
                {ER-AltDef}
    \and
    \EmissionRuleL{\hspace{1.1cm} \Alt{r}{alt}{tag} \\\\
                  \AltParameter{r}{tag}{\pi(x)}{x}}
                {\pilcode{r <- case s of} \\\\
                  \\ \pilcode{(tag ... x ...) @ alt -> ...}}
                {ER-AltNode}
    \end{mathpar}
  \caption{Emission rules for case expressions}
  \label{fig:er-case}
  \end{figure}

	\subsection{Function definitions}
	
	\textsc{ER-Fun} relates function parameters to the function's name, and \textsc{ER-FunRet} relates the return value of a binding sequence to the encompassing function's name. \textsc{ER-Ext} handles external declarations, and \textsc{ER-Entry} denotes the main entry point of the program.

	\begin{figure}[h]
		\begin{mathpar}
    \EmissionRule{\FunctionParameter{f}{\pi(x)}{x}}
               {\pilcode{f ... x ... = ...}}
               {ER-Fun}
    \and
    \EmissionRuleL{\ReturnValue{f}{x}}
                {\pilcode{f ... =} \\\\
                \\ \dots \\\\
                \\ \pilcode{pure x}}
                {ER-FunRet}
    \and
    \EmissionRuleL{\External{f}{\epsilon(effect)}{retTy} \\\\
                  \ExternalParam{f}{\pi(ty)}{ty}}
                {\pilcode{external <effect>} \\\\
                \\ \pilcode{f :: ... -> ty -> ... -> retTy}}
                {ER-Ext}
    \and
    \EmissionRule{\EntryPoint{main}}
                { }
                {ER-Entry}
		\end{mathpar}
		\caption{Emission rules for functions}
		\label{fig:er-fundefs}
	\end{figure}


  \section{Points-to analysis}
  
  Points-to analyses are a standard type of data-flow analyses. They build an abstract representation of the heap layout of the program (abstract store) by establishing a mapping between abstract heap locations and the possible values they might contain. For GRIN, the \emph{heap points-to analysis} is a path sensitive, but calling context insensitive points-to analysis originally described in \cite{boquist-phd}. Here, we give a logic defined formalization of the analysis in Datalog, concisely expressed in terms of the CreatedBy analysis (see Section~\ref{sec:created-by}). It is important to note, that this version of the analysis only builds the abstract store (locations $\rightarrow$ values mapping) and not the abstract environment (variable $\rightarrow$ values mapping). This is because in order to define the liveness analysis, it is sufficient to calculate only the abstract store. In Figure~\ref{fig:heap-analysis} we can see the Datalog rules of the heap analysis.

  \begin{figure}[h]
    \begin{mathpar}
      \InferRule{\Store{v}{i}}
                {\Heap{v}{i}}
                {H-Store}
      \and
      \InferRule{\Update{\any}{p}{val} \\\\
                \CreatedBy{p}{p'} \\\\
                \Heap{p'}{\any}}
                {\Heap{p'}{val}}
                {H-Update}
    \end{mathpar}

  \caption{Rules of the Heap analysis}
  \label{fig:heap-analysis}
  \end{figure}

  Rule \textsc{H-Store} creates a new abstract location for each \pilcode{store} instruction in the GRIN program. This means, the number of abstract locations the analysis keeps track of is statically bounded. In fact it is equal to the number of \pilcode{store} instructions in the program, implying that the heap analysis over-approximates the actual heap layout. This approximation is always safe, which means a location will always be mapped to a set of statically computed values that is a superset of all the possible dynamically occurring values. Note, that the abstract locations are identified by the pointer's name the \pilcode{store} instruction returns.

	Rule \textsc{H-Update} extends the abstract locations with all the possible values they might be updated with. This is the rule where the complexity comes in due to the \pilcode{CreatedBy} relation. This relation essentially tracks back the pointer to be updated to its origin. This is necessary, since pointers can alias each other, and we must trace them back to the location they point to.
	
	The \any symbol in the relations mean that the given argument is irrelevant. Any value can fill that argument, and the it won't be used in the rule.

  \section{Created-by analysis} \label{sec:created-by}
  
  The created-by analysis answers the question: \emph{What program point was a given variable created by?}. It traces back variables to their possible origins. 
  For example a variable might have been an alt parameter inside a case expression scrutinizing a node fetched from the heap put their by a function, and so on. We would like to trace back such variables to their \emph{original} definition sites. It is important to mention that a variable may have multiple possible origins since the control-flow can branch in case expressions. For instance, the result of a case expression could have been created in any of the alternatives.
  
  In this section, we present a Datalog formalization of a path sensitive and calling context insensitive created-by analysis for GRIN.
  
  The created-by analysis was originally described by Remi Turk \cite{remi-masters} who introduced it as an extended points-to analysis. 
  
  \subsection{Origin rules}
  
  The rules in Figure~\ref{fig:cby-origins} are the origin rules of the created-by analysis. The premises of the rules are all and the only possible ways a value can be created. This means that all variables can be traced back to a binding that has a literal assignment, node value creation, \pilcode{store}, \pilcode{update} or external functional call right-hand side. These origins are also called \emph{producers} in the context of dead data elimination (see Section~\ref{sec:dde}).

  \begin{figure}[h]
    \begin{mathpar}
    \InferRule{\LitAssign{v}{\any}{\any}}
              {\CreatedBy{v}{v}}
              {CBy-Lit}
    \and
    \InferRule{\Node{n}{\any}}
              {\CreatedBy{n}{n}}
              {CBy-Node}
    \and
    \InferRule{\Store{p}{\any}}
              {\CreatedBy{p}{p}}
              {CBy-Store}
    \and
    \InferRule{\Update{v}{\any}{\any}}
              {\CreatedBy{v}{v}}
              {CBy-Update}
    \and
    \InferRule{\Call{v}{f} \\\\
    					\External{f}{\any}{\any}}
              {\CreatedBy{v}{v}}
              {CBy-Ext}
    \end{mathpar}
  \caption{Origin rules of the created-by analysis}
  \label{fig:cby-origins}
  \end{figure}

	\subsection{Non-case related rules}
	
	Figure~\ref{fig:cby-noncase} shows all the recursive rules of the created-by analysis that do not deal with case expressions. \textsc{CBy-Move} is a transitive relation, \textsc{CBy-FunParam} traces back origin through function parameters, \textsc{CBy-Fetch} handles \pilcode{fetch} instructions, \textsc{CBy-Call} propagates information \emph{into} function parameters, \textsc{CBy-FunRet} propagates information \emph{out} of function return values, and finally \textsc{CBy-AsPatNode} and \textsc{CBy-AsPatVar} flow information through the node pattern and variable parts of @patterns. 

  \begin{figure}[h]
    \begin{mathpar}
    \InferRule{\Move{v}{n} \\\\
                \CreatedBy{n}{n'}}
                {\CreatedBy{v}{n'}}
                {CBy-Move}
		\and
		\InferRule{\FunctionParameter{\any}{\any}{p} \\\\
							\CreatedBy{v_1}{p} \\\\
							\CreatedBy{p}{v_2}}
							{\CreatedBy{v_1}{v_2}}
							{CBy-FunParam}
		\and
		\DatalogRule{\CreatedBy{v}{n'}}
								{\Fetch{v}{p} \\\\
									\CreatedBy{p}{p'} \\\\
									\Heap{p'}{n} \\\\
									\CreatedBy{n}{n'}}
									{CBy-Fetch}
		\and
		\DatalogRule{\CreatedBy{x}{arg'}}
								{\CallArgument{r}{i}{arg} \\\\
								\Call{r}{f} \\\\
								\FunctionParameter{f}{i}{x} \\\\
								\CreatedBy{arg}{arg'}}
								{CBy-Call}
		\and
		\DatalogRule{\CreatedBy{r}{v'}}
								{\Call{r}{f} \\\\
								\ReturnValue{f}{v} \\\\
								\CreatedBy{v}{v'}}
								{CBy-FunRet}
		%TODO: NodeCreatedBy
		\and
		\DatalogRule{\CreatedBy{u}{arg'}}
								{\NodePattern{v}{tag}{n} \\\\
								\NodeParameter{v}{i}{u} \\\\
								\CreatedBy{n}{n'} \\\\
								\Node{n'}{tag} \\\\
								\NodeArgument{n'}{i}{arg} \\\\
								\CreatedBy{arg}{arg'}}
								{CBy-AsPatNode}
		\and
		\DatalogRule{\CreatedBy{v}{n'}}
								{\NodePattern{v}{\any}{n} \\\\
								\CreatedBy{n}{n'}}
								{CBy-AsPatVar}
    \end{mathpar}
  \caption{Non-case related rules of the created-by analysis}
  \label{fig:cby-noncase}
  \end{figure}

	\subsection{Case expression related rules} \label{subsec:cby-case}
	
	In Figure~\ref{fig:cby-case} we can see all the rules of thecreated-by analysis related to case expressions and the case alternatives. Rules \textsc{CBy-AltLit}, \textsc{CBy-AltNode} and \textsc{CBy-AltDef} propagate information between the names of (literal, node and default) alternatives and the case expression's scrutinee. That is, the origin of the alternative's name is the same as the origin of the scrutinee. The analysis is path-sensitive, so the \textsc{CBy-AltNode} rule only propagates information from the scrutinee to the alternative if the alternative is possible during the execution of the program. In other words, it must be checked whether the scrutinee's origin can be a node value with the same tag as the node in the alternative's pattern. This is a safe static approximation of the runtime behaviour. Literal-valued and default alternatives are always considered possible.
	
	\textsc{CBy-AltParam} propagates origin information from the scrutinee's node arguments into the alternative's paramaters if the tags of the scrutinee's node value and the alternative's pattern are the same.
	
	The rules \textsc{CBy-AltNode-Ret}, \textsc{CBy-AltLit-Ret} and \textsc{CBy-AltDef-Ret} propagate the origins of the alternatives' return values into the result of the case expression, if the given alternative is possible.

  \begin{figure}[h]
    	\begin{mathpar}
      \DatalogRule{\CreatedBy{alt}{scrut'}}
                  {\Case{r}{scrut} \\\\
                  \AltLiteral{r}{alt}{\any} \\\\
                  \CreatedBy{scrut}{scrut'}}
                  {CBy-AltLit}
      \and
      \DatalogRule{\CreatedBy{alt}{scrut'}}
                  {\Case{r}{scrut} \\\\
                  \Alt{r}{alt}{tag} \\\\
                  \CreatedBy{scrut}{scrut'} \\\\
                  \Node{scrut'}{tag}}
                  {CBy-AltNode}
      \and
      \DatalogRule{\CreatedBy{alt}{scrut'}}
                  {\Case{r}{scrut} \\\\
                  \AltDefault{r}{alt} \\\\
                  \CreatedBy{scrut}{scrut'}}
                  {CBy-AltDef}
      \and
      % TODO: shorter name for altParam
      % TODO: shorter name for nodeArg
      % TODO: PossibleAlt
      \DatalogRule{\CreatedBy{altParam}{nodeArg'}}
                  {\Case{r}{scrut} \\
                  \Alt{r}{\any}{tag} \\\\
                  \AltParameter{r}{tag}{i}{altParam} \\\\
                  \CreatedBy{scrut}{scrut'} \\
                  \Node{scrut'}{tag} \\\\
                  \NodeArgument{scrut'}{i}{nodeArg} \\\\
                  \CreatedBy{nodeArg}{nodeArg'}}
                  {CBy-AltParam}
      \and
      \DatalogRule{\CreatedBy{r}{v'}}
                  {\Case{r}{scrut} \\
                  \Alt{r}{alt}{tag} \\\\
                  \CreatedBy{scrut}{scrut'} \\
                  \Node{scrut'}{tag} \\\\
                  \ReturnValue{alt}{v} \\
                  \CreatedBy{v}{v'}}
                  {CBy-AltNode-Ret}
      % TODO: maybe OR?
      \and
      \DatalogRule{\CreatedBy{r}{v'}}
                  {\Case{r}{\any} \\\\
                  \AltLiteral{r}{alt}{\any} \\\\
                  \ReturnValue{alt}{v} \\\\
                  \CreatedBy{v}{v'}}
                  {CBy-AltLit-Ret}
      \and
      \DatalogRule{\CreatedBy{r}{v'}}
                  {\Case{r}{\any} \\\\
                  \AltDefault{r}{alt} \\\\
                  \ReturnValue{alt}{v} \\\\
                  \CreatedBy{v}{v'}}
                  {CBy-AltDef-Ret}
    \end{mathpar}
  \caption{Case-related rules of the created-by analysis}
  \label{fig:cby-case}
  \end{figure}

  \section{Liveness analysis} \label{sec:lva}

	Liveness analysis is a classic backwards data-flow analysis which determines which variables might be live during the execution of the program. Here, we present a path-sensitive and calling context insensitive Datalog formalization of the liveness analysis for GRIN. The analysis computes liveness information, for simple values, function parameters, function return values, node arguments on the stack and node arguments on the heap as well. The different types of liveness information is captured by different liveness relations. This level of granularity is required by the dead data elimination transformation (see Section~\ref{sec:dde}).
	
	\subsection{Helper relations}
	
	In order to ease the complexity of the liveness analysis we introduce some helper relations. The rules for these relations can be seen in Figure~\ref{fig:lva-helpers}. Most of the rules are straightforward, however it is worth mentioning that the path-sensitivity checks for case alternatives are factored out into the \textsc{PA-Def}, \textsc{PA-Lit} and \textsc{PA-Node} rules.

  \begin{figure}[h]

    \begin{mathpar}
		\DatalogRule{\NodeCreatedBy{n}{t}{n'}}
								{\Node{n'}{t} \\\\
								\CreatedBy{v}{v'}}
								{NCBy}
		\and
		\DatalogRule{\PossibleNodeTag{n}{t}}
								{\NodeCreatedBy{n}{t}{\any}}
								{PNT}
		\and
		\DatalogRule{\PossibleAlt{alt}}
								{\AltDefault{\any}{alt}}
								{PA-Def}
		\and
		\DatalogRule{\PossibleAlt{alt}}
								{\AltLiteral{\any}{alt}{\any}}
								{PA-Lit}
		\and
		\DatalogRule{\PossibleAlt{alt}}
								{\Case{r}{scrut} \\\\
								\Alt{r}{alt}{t} \\\\
								\NodeCreatedBy{scrut}{t}{scrut'}}
								{PA-Node}
		\and
		\DatalogRule{\PointerOrigin{p}{q}}
								{\Store{q}{\any} \\\\
								\CreatedBy{p}{q}}
								{PO}
		\and
		\DatalogRule{\PossibleLocTag{p}{t}}
								{\Heap{p}{n} \\\\
								\PossibleNodeTag{n}{tag}}
								{PLT}
	  \end{mathpar}
  \caption{Helper relations for the liveness analysis}
  \label{fig:lva-helpers}
  \end{figure}

	\section{Liveness of simple values}
	
	The rules in Figures~\ref{fig:lva-roots}~and~\ref{fig:lva-simple-vals} describe the liveness data-flow for simple values. If GRIN only had simple values, these rules would be sufficient to calculate the required liveness information. The two root rules, \textsc{LS-Entry} and \textsc{Ext} mark the liveness roots. The roots are the return value of the main entry point, and any arguments to a side-effecting external call. From these roots, the liveness is propagated backwards in the program through rules like \textsc{LS-Move}.

  \begin{figure}[h]
    	\begin{mathpar}
      \DatalogRule{\LiveSVal{x}}
                  {\EntryPoint{main} \\\\
                  \ReturnValue{main}{x}}
                  {LS-Entry}
      \and
      \DatalogRule{\LiveSVal{x}}
                  {\Call{y}{f} \\\\
                  \CallArgument{y}{\any}{x} \\\\
                  \External{f}{true}{\any}}
                  {LS-Ext}
      \end{mathpar}
  \caption{Root liveness rules}
  \label{fig:lva-roots}
  \end{figure}

  The \textsc{LS-AltRet} rule propagates simple liveness from the result of the case expression to the return value of the alternative, given that the alternative is possible. \textsc{LS-Scrut} is also a rule related to case expressions. It propagates simple liveness information from the alternative back to the scrutinee.
  
  Rules \textsc{LS-FunRet} and \textsc{LFS-FunRet} propagate simple liveness from the result of a function call back to the function's return value. The observant reader might notice that these relation could have been expressed by using solely the \textsc{ReturnValue} relation. However, this is only true for regular functions. Since external functions do not have definitions, we need to introduce a special relation, \textsc{LiveFunRetSimple} that captures the liveness of their return values.

	\begin{figure}[h]
			\begin{mathpar}
      \DatalogRule{\LiveSVal{x}}
              {\LiveSVal{y} \\\\
              \Move{y}{x}}
              {LS-Move}
      \and
      % TODO: alt helper relation
      \DatalogRule{\LiveSVal{ret}}
                  {\LiveSVal{res} \\ \PossibleAlt{alt} \\ \ReturnValue{alt}{ret} \\\\
                  (\Alt{res}{alt}{\any} \lor \AltLiteral{res}{alt}{\any} \lor \AltDefault{res}{alt})}
                  {LS-AltRet}
      \and
      \DatalogRule{\LiveSVal{scrut}}
                  {\LiveSVal{alt} \\ \PossibleAlt{alt} \\ \Case{res}{scrut} \\\\
                  (\Alt{res}{alt}{\any} \lor \AltLiteral{res}{alt}{\any} \lor \AltDefault{res}{alt})}
                  {LS-Scrut}
      \and
      \DatalogRule{\LiveFunRetSimple{f}}
                  {\LiveSVal{y} \\\\
                  \Call{y}{f}}
                  {LFS-FunRet}
      \and
      \DatalogRule{\LiveSVal{ret}}
                  {\LiveFunRetSimple{f} \\\\
                  \ReturnValue{f}{ret}}
                  {LS-FunRet}
      \and
      \DatalogRule{\LiveSVal{x}}
                  {\LiveSVal{param} \\\\
                  \FunctionParameter{f}{i}{param} \\\\
                  \Call{y}{f} \\
                  \CallArgument{y}{i}{x}}
                  {LS-CallArg}
      \and
      \DatalogRule{\LiveSVal{x}}
                  {\LiveFunRetSimple{f_{ext}} \\\\
                  \External{f_{ext}}{\any}{\any} \\\\
                  \Call{y}{f_{ext}} \\\\
                  \CallArgument{y}{\any}{x}}
                  {LS-ExtArg}
      \and
      \DatalogRule{\LiveSVal{x}}
                  {\LiveNodeArg{n}{t}{i} \\\\
                  \Node{n}{t} \\\\
                  \NodeArgument{n}{i}{x}}
                  {LS-NodeArg}
	  \end{mathpar}
  \caption{Liveness rules for simple values}
  \label{fig:lva-simple-vals}
  \end{figure}

	The rules \textsc{LS-CallArg} and \textsc{LS-ExtArg} propagate simple liveness through regular function call arguments and external function call arguments respectively.
	
	Finally, rule \textsc{LS-NodeArg} propagates the node argument's liveness back to the simple value.
	
	\subsection{Liveness of node values}
	
	Figure~ref{fig:lva-nodes} shows the rules for the liveness analysis of node values on the stack. \textsc{LNA-AsPat} propagates the liveness of node arguments from the @pattern's parameters to the node with the same tag contained in the variable on the right-hand side of the binding. On the other hand, \textsc{LNA-AsPat-Alias} propagates all node argument liveness from the alias of the @pattern back to the variable on the right-hand side. \textsc{LNA-Move} is very similar to \textsc{LNA-AsPat-Alias}, just for variable copying.

  \begin{figure}[h]
    \begin{mathpar}
    \DatalogRule{\LiveNodeArg{n}{t}{i}}
								{\LiveSVal{x} \\\\
								\NodeParameter{as}{i}{x} \\\\
								\NodePattern{as}{t}{n}}
								{LNA-AsPat}
		\and
		\DatalogRule{\LiveNodeArg{n}{t}{i}}
								{\LiveNodeArg{as}{t}{i} \\\\
								\NodePattern{as}{t}{n}}
								{LNA-AsPat-Alias}
		\and
		\DatalogRule{\LiveNodeArg{m}{t}{i}}
								{\LiveNodeArg{n}{t}{i} \\\\
								\Move{n}{m}}
								{LNA-Move}
    \end{mathpar}
  \caption{Liveness rules for node values}
  \label{fig:lva-nodes}
  \end{figure}

	\subsection{Liveness of node-valued case expressions}
	
	The rules in Figure~\ref{fig:lva-case} show how liveness information flows through nodes in case expressions. \textsc{LNA-Scrut-Name} describes the node liveness data-flow from the alternative's name back into the scrutinee. \textsc{LNA-Scrut-Pat} is similar, but the source of liveness is the alternative's node pattern instead.
	
	\textsc{LNA-AltNode} propagates node liveness from the result of the case expression into the alternative's return value, given that the alternative is possible and that the result and the return value have a common a tag. \textsc{LNA-AltLitDef} is the analogue rule for alternatives with literal and default patterns.

  \begin{figure}[h]
    \begin{mathpar}
    \DatalogRule{\LiveNodeArg{scrut}{t}{i}}
								{\LiveNodeArg{alt}{t}{i} \\\\
								\Case{r}{scrut} \\
								\Alt{r}{alt}{t} \\
								\AltParameter{r}{t}{i}{\any}}
								{LNA-Scrut-Name}
		\and
		\DatalogRule{\LiveNodeArg{scrut}{t}{i}}
								{\LiveSVal{x} \\\\
								\Case{r}{scrut} \\
								\AltParameter{r}{t}{i}{x}}
								{LNA-Scrut-Pat}
		\and
		\DatalogRule{\LiveNodeArg{ret}{t_{res}}{i}}
								{\LiveNodeArg{res}{t_{res}}{i} \\\\
								\Case{res}{scrut} \\
								\Alt{res}{alt}{t_{alt}} \\
								\ReturnValue{alt}{ret} \\\\
								\PossibleAlt{alt} \\
								\NodeCreatedBy{ret}{t_{res}}{ret'}}
								{LNA-AltNode}
		\and
		\DatalogRule{\LiveNodeArg{ret}{t_{res}}{i}}
								{\LiveNodeArg{res}{t_{res}}{i} \\\\
								\Case{res}{\any} \\
								(\AltLiteral{res}{alt}{\any} \lor \AltDefault{res}{alt}) \\\\
								\ReturnValue{alt}{ret} \\
								\NodeCreatedBy{ret}{t_{res}}{ret'}}
								{LNA-AltLitDef}
    \end{mathpar}
  \caption{Liveness rules for node-valued case expressions}
  \label{fig:lva-case}
  \end{figure}

	\subsection{Liveness of node-valued functions}

	In Figure~\ref{fig:lva-function} we can see the different rules for propagating node argument liveness \emph{into} functions with node-valued return values (\textsc{LFN}), \emph{out of} functions with node-valued return values (\textsc{LNA-FunRet}) and \emph{out of} functions with node-valued parameters (\textsc{LNA-Call}).

  \begin{figure}[h]
    \begin{mathpar}
    \DatalogRule{\LiveFunRetNodeArg{f}{t}{i}}
								{\LiveNodeArg{y}{t}{i} \\\\
								\Call{y}{f}}
								{LFN}
		\and
		\DatalogRule{\LiveNodeArg{r}{t}{i}}
								{\LiveFunRetNodeArg{f}{t}{i} \\\\
								\ReturnValue{f}{r}}
								{LNA-FunRet}
		\and
		\DatalogRule{\LiveNodeArg{x}{t}{i}}
								{\LiveNodeArg{param}{t}{i} \\\\
								\FunctionParameter{f}{j}{param} \\\\
								\Call{y}{f} \\
								\CallArgument{y}{j}{x}}
								{LNA-Call}
    \end{mathpar}
  \caption{Liveness rules for node-valued functions}
  \label{fig:lva-function}
  \end{figure}

	\subsection{Liveness of heap objects}
	
	The rules in Figure~\ref{fig:lva-heap} describe how the liveness of node values on the heap are computed. Since information flows backwards, the \pilcode{fetch} operation propagates node argument liveness \emph{onto} the heap as shown by rule \textsc{LHN}. Similarly, \pilcode{store} propagates node liveness \emph{from} the heap into the variable on the left-hand side (\textsc{LNA-Store}), just like the \pilcode{update} operation which also flows liveness \emph{from} the heap into the variable used for updating (\textsc{LNA-Update}).
	
  \begin{figure}[h]
    \begin{mathpar}
    \DatalogRule{\LiveHeapNodeArg{q}{t}{i}}
								{\LiveNodeArg{n}{t}{i} \\\\
								\Fetch{n}{p} \\\\
								\PointerOrigin{p}{q} \\\\
								\PossibleLocTag{q}{t}}
								{LHN}
		\and
		\DatalogRule{\LiveNodeArg{n}{t}{i}}
								{\LiveHeapNodeArg{p}{t}{i} \\\\
								\Store{q}{n} \\\\
								\PointerOrigin{p}{q} \\\\
								\PossibleNodeTag{n}{t}}
								{LNA-Store}
		\and
		\DatalogRule{\LiveNodeArg{n}{t}{i}}
								{\LiveHeapNodeArg{q}{t}{i} \\\\
								\Update{\any}{p}{n} \\\\
								\PointerOrigin{p}{q} \\\\
								\PossibleNodeTag{n}{t}}
								{LNA-Update}
    \end{mathpar}
  \caption{Liveness rules for heap operations}
  \label{fig:lva-heap}
  \end{figure}



\end{document}
