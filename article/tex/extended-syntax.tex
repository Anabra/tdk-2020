\documentclass[main.tex]{subfiles}
\begin{document}

  In this chapter we present our first contribution, a new syntactical structure for the GRIN IR.

  \section{Motivation}

  \section{Extended syntax}
  
	\subfile{syntax}

  \section{The small example revisited}

  \hspace{-0.5cm}
  \begin{codeFloat}
    \begin{minipage}{0.48\textwidth}
      \begin{haskell}
        grinMain =
          k1 <- pure 1
          n1 <- pure (CInt k1)
          a  <- %\textcolor{identifierColor}{store}% n1
          k2 <- pure 2
          n2 <- pure (CInt k2)
          b  <- %\textcolor{identifierColor}{store}% n2
          k3 <- pure 3
          n3 <- pure (CInt k3)
          c  <- %\textcolor{identifierColor}{store}% n3

          n4  <- pure (FAdd b c)
          r   <- %\textcolor{identifierColor}{store}% n4
          n5  <- pure (P1_add a)
          suc <- pure n5
          _1  <- apply suc r

        add x y =
          x' <- eval x
          (CInt x1) @ _2 <- pure x'
          y' <- eval y
          (CInt y1) @ _3 <- pure y'
          add.r <- _prim_int_add x1 y1
          n6 <- pure (CInt add.r)
          pure n6
      \end{haskell}
    \end{minipage}
    \hfill
    \begin{minipage}{0.50\textwidth}
      \begin{haskellNum}{12}
       eval p =
         v <- %\textcolor{identifierColor}{fetch}% p
         eval.r <- case v of
           (CInt _n) @ alt1 ->
              pure v
           (P2_add) @ alt2 ->
              pure v
           (P1_add _x) @ alt3 ->
              pure v
           (Fadd x2 y2) @ alt4 ->
             r_add <- add x2 y2
             _4 <- %\textcolor{identifierColor}{update}% p r_add
             pure r_add
         pure eval.r


       apply f u =
         apply.r <- case f of
           (P2_add) @ alt5 ->
             f' <- pure (P1_add u)
             pure f'
           (P1_add z) @ alt6 ->
             res <- add z u
             pure res
         pure apply.r
      \end{haskellNum}
    \end{minipage}
    \caption{Extended GRIN code generated from \pilcode{(add 1) (add 2 3)}}
    \label{code:grin-add-es}

  \end{codeFloat}

\end{document}
